%
% This is a borrowed LaTeX template file for lecture notes for CS267,
% Applications of Parallel Computing, UCBerkeley EECS Department.
% Now being used for CMU's 10725 Fall 2012 Optimization course
% taught by Geoff Gordon and Ryan Tibshirani.  When preparing 
% LaTeX notes for this class, please use this template.
%
% To familiarize yourself with this template, the body contains
% some examples of its use.  Look them over.  Then you can
% run LaTeX on this file.  After you have LaTeXed this file then
% you can look over the result either by printing it out with
% dvips or using xdvi. "pdflatex template.tex" should also work.
%

\documentclass[twoside]{article}


\setlength{\oddsidemargin}{0.25 in}
\setlength{\evensidemargin}{-0.25 in}
\setlength{\topmargin}{-0.6 in}
\setlength{\textwidth}{6.5 in}
\setlength{\textheight}{8.5 in}
\setlength{\headsep}{0.75 in}
\setlength{\parindent}{0 in}
\setlength{\parskip}{0.1 in}

%
% ADD PACKAGES here:
%

\usepackage{amsmath,amsfonts,graphicx}
\usepackage{booktabs}
\usepackage{multirow}
\usepackage{xcolor}
\usepackage{float}

%
% The following commands set up the lecnum (lecture number)
% counter and make various numbering schemes work relative
% to the lecture number.
%
\newcounter{lecnum}
\renewcommand{\thepage}{\thelecnum-\arabic{page}}
\renewcommand{\thesection}{\thelecnum.\arabic{section}}
\renewcommand{\theequation}{\thelecnum.\arabic{equation}}
\renewcommand{\thefigure}{\thelecnum.\arabic{figure}}
\renewcommand{\thetable}{\thelecnum.\arabic{table}}

%
% The following macro is used to generate the header.
%
\newcommand{\lecture}[4]{
   \pagestyle{myheadings}
   \thispagestyle{plain}
   \newpage
   \setcounter{lecnum}{#1}
   \setcounter{page}{1}
   \noindent
   \begin{center}
   \framebox{
      \vbox{\vspace{2mm}
    \hbox to 6.28in { {\bf UCSB CS 291D: Blockchains and Cryptocurrencies
	\hfill Fall 2020} }
       \vspace{4mm}
       \hbox to 6.28in { {\Large \hfill Lecture #1: #2  \hfill} }
       \vspace{2mm}
       \hbox to 6.28in { {\it Lecturer: #3 \hfill Scribes: #4} }
      \vspace{2mm}}
   }
   \end{center}
   \markboth{Lecture #1: #2}{Lecture #1: #2}

%   {\bf Note}: {\it LaTeX template courtesy of UC Berkeley EECS dept.}

%   {\bf Disclaimer}: {\it These notes have not been subjected to the
%   usual scrutiny reserved for formal publications.  They may be distributed
%   outside this class only with the permission of the Instructor.}
%   \vspace*{4mm}
}
%
% Convention for citations is authors' initials followed by the year.
% For example, to cite a paper by Leighton and Maggs you would type
% \cite{LM89}, and to cite a paper by Strassen you would type \cite{S69}.
% (To avoid bibliography problems, for now we redefine the \cite command.)
% Also commands that create a suitable format for the reference list.
\renewcommand{\cite}[1]{[#1]}
\def\beginrefs{\begin{list}%
        {[\arabic{equation}]}{\usecounter{equation}
         \setlength{\leftmargin}{2.0truecm}\setlength{\labelsep}{0.4truecm}%
         \setlength{\labelwidth}{1.6truecm}}}
\def\endrefs{\end{list}}
\def\bibentry#1{\item[\hbox{[#1]}]}

%Use this command for a figure; it puts a figure in wherever you want it.
%usage: \fig{NUMBER}{SPACE-IN-INCHES}{CAPTION}
\newcommand{\fig}[3]{
			\vspace{#2}
			\begin{center}
			Figure \thelecnum.#1:~#3
			\end{center}
	}
% Use these for theorems, lemmas, proofs, etc.
\newtheorem{theorem}{Theorem}[lecnum]
\newtheorem{lemma}[theorem]{Lemma}
\newtheorem{proposition}[theorem]{Proposition}
\newtheorem{claim}[theorem]{Claim}
\newtheorem{corollary}[theorem]{Corollary}
\newtheorem{definition}[theorem]{Definition}
\newenvironment{proof}{{\bf Proof:}}{\hfill\rule{2mm}{2mm}}

% **** IF YOU WANT TO DEFINE ADDITIONAL MACROS FOR YOURSELF, PUT THEM HERE:

\newcommand\E{\mathbb{E}}

\begin{document}
%FILL IN THE RIGHT INFO.
%\lecture{**LECTURE-TITLE**}{**DATE**}{**LECTURER**}{**SCRIBE**}
\lecture{5}{Consensus III: Speed Up Consensus using Randomness}{Derek Leung}{Dipanjan Das, Chinmay Sonar}
%\footnotetext{These notes are partially based on those of Nigel Mansell.}

% **** YOUR NOTES GO HERE:

% Some general latex examples and examples making use of the
% macros follow.  
%**** IN GENERAL, BE BRIEF. LONG SCRIBE NOTES, NO MATTER HOW WELL WRITTEN,
%**** ARE NEVER READ BY ANYBODY.

\section{Desirable properties of a cryptocurrency system} % Don't be this informal in your notes!
For a cryptocurrency system to be practical and usable, it should meet the following requirements:

\begin{itemize}
	\item \textbf{Secure:}
	\begin{itemize}
		\item \textbf{Theft:} Only owner should be able to spend the money, and no one else.
		\item \textbf{Consensus:} If money is spent, everybody should agree.
		In other words, all the parties must agree to the order of transactions that have taken place in the system.
		Also, no money can be double spent.
	\end{itemize}
	\item \textbf{Distributed:} The transaction log is distributed among all the parties using the system.
	\item \textbf{Anonymity:} Only transactions are public, but not the identities of the sender, or the receiver.
	\item \textbf{Scalable:} Achieves high throughput in terms of the number of transactions executed per unit time.
	\item \textbf{Efficient:} The storage space required to store the transaction log, and the bandwidth and the processing power required to execute the distributed protocol should be optimal.
	\item \textbf{Fault tolerance:} Should be resilient to network interruptions, or faulty, or adversarial nodes.
	\item \textbf{Fairness:} Like in any trade, the payer receives~\cite{1} the item/service she has paid for.
\end{itemize}


\section{The Byzantine Generals Problem (BGP) -- Revisited}
The Byzantine Generals Problem allegorically models a fundamental problems in the distributed systems where $n$ nodes need to come to a consensus under the following assumptions:

\begin{itemize}
	\item \textbf{Authenticated channel:} Receivers know the source of the message they receive.
	\item \textbf{Bad actors:} There are $f$ bad actors present in the system who try to either break the consistency of the system, or its availability.
	\item \textbf{Delay:} Message delays are unbounded, \textit{i.e.}, the network may be unreliable.
\end{itemize}

\begin{theorem}
	It is impossible to solve the problem unless $3f + 1 \leq n$.
	In other words, in a synchronous network with authenticated channel where $f$ actors are corrupt, no solution will work unless there are more than 3f actors present in the network. 
\end{theorem}

\subsection{Proving impossibility by example}
Let's assume $n=3$ actors with $f=1$ corrupt actor want to agree on a bit $b \in \{0,1\}$.
Table~\ref{tab:bgp_impossbility_proof1} and Table~\ref{tab:bgp_impossbility_proof2} show the bits sent and received by different actors.
Actor \textbf{B} is dishonest.
In the first case, it sends bit $0$ to \textbf{A}, and bit $1$ to \textbf{C}.
Therefore, \textbf{A} and \textbf{C} agree on different bits---$0$ and $1$ respectively.
In the seconds case, it sends no message at all---thus preventing \textbf{A} and \textbf{C} reach any consensus.
Either one breaks the consistency of the system.

\begin{minipage}{.45\textwidth} %
\begin{table}[H]
	\centering
	\renewcommand{\arraystretch}{1.5}
	\begin{tabular}{cc|ccc}
		&  & \multicolumn{3}{c}{\textbf{Sent}} \\ \cline{3-5} 
		& \textbf{} & \textbf{A} & \multicolumn{1}{c}{\textbf{B}} & \multicolumn{1}{c}{\textbf{C}} \\ \midrule
		\multicolumn{1}{l|}{\multirow{3}{*}{\rotatebox{90}{\textbf{Received}}}} & \textbf{A} & 1 & \textcolor{red}{0} & 0 \\
		\multicolumn{1}{l|}{} & \textbf{B} & 1 & --- & 0 \\
		\multicolumn{1}{l|}{} & \textbf{C} & 1 & \textcolor{red}{1} & 0 \\ 
	\end{tabular}
	\caption{Impossibility of the Byzantine Generals Problem with less than $3f + 1$ actors.
		\textbf{B}, who sends two different messages to \textbf{A} and \textbf{C}, is the bad actor here.}
	\label{tab:bgp_impossbility_proof1}
\end{table}
\end{minipage} %
\hfill
\begin{minipage}{.45\textwidth} %
\begin{table}[H]
	\centering
	\renewcommand{\arraystretch}{1.5}
	\begin{tabular}{cc|ccc}
		&  & \multicolumn{3}{c}{\textbf{Sent}} \\ \cline{3-5} 
		& \textbf{} & \textbf{A} & \multicolumn{1}{c}{\textbf{B}} & \multicolumn{1}{c}{\textbf{C}} \\ \midrule
		\multicolumn{1}{l|}{\multirow{3}{*}{\rotatebox{90}{\textbf{Received}}}} & \textbf{A} & 1 & \textcolor{red}{---} & 0 \\
		\multicolumn{1}{l|}{} & \textbf{B} & 1 & --- & 0 \\
		\multicolumn{1}{l|}{} & \textbf{C} & 1 & \textcolor{red}{---} & 0 \\ 
	\end{tabular}
	\caption{Impossibility of the Byzantine Generals Problem with less than $3f + 1$ actors.
		\textbf{B}, who sends no message to \textbf{A} and \textbf{C}, is the bad actor here.}
	\label{tab:bgp_impossbility_proof2}
\end{table}
\end{minipage}


\subsection{Practical Byzantine Fault Tolerance (PBFT)}
Practical Byzantine Fault Tolerance (PBFT) is a leader based, two-round protocol for the Byzantine Generals Problem.
Let's assume there are $n$ actors of which $f$ are corrupt, where $3f + 1 \leq n$.
The steps of the algorithm are:

\begin{enumerate}
	\item Some leader proposes a transaction $T$ by broadcasting it to others
	\item When machine $x$ hears the transaction $T$, it acknowledges the receipt by broadcasting $\texttt{ACK}(x, T)$
	\item When machine $y$ hears quorum of $\texttt{ACK}(x, T)$, it acknowledges the receipt by broadcasting $\texttt{ACK}(y, \text{quorum of } \texttt{ACK}(x, T))$.
	A \textit{quorum} is defined as $2f + 1$ acknowledgment messages $\texttt{ACK}(x, T)$ for the same transaction $T$
	\item When machine $z$ hears $\texttt{ACK}(y, \text{quorum of } \texttt{ACK}(x, T))$, it knows $T$ is committed
\end{enumerate}


\subsection{PBFT by example}
Let's assume there are $n=4$ actors out of which $f=1$ are corrupt.
Therefore, quorum ($2f + 1$) is $3$ messages/actor.
There actors are $\{x-1, x_2, x_3, x_4\}$.
We assume the leader is \textit{not} dishonest.
Let $x_2$ be the bad actor here.

\begin{enumerate}
	\item \textbf{Round 0 (Proposal): } $x_2$ proposes a transaction $T$.
	$x_i$, where $i \in \{1, 3, 4\}$, listens to this broadcast
	\item \textbf{Round 1: } $x_i$ acknowledges the receipt by broadcasting $\texttt{ACK}(x_i, T)$, where $i \in \{1, 3, 4\}$
	\item \textbf{Round 2: } $x_i$ re-acknowledges the acknowledgment messages from the round $1$ by broadcasting $\texttt{ACK}(x_i, \texttt{ACK}(\{x_j\}, T))$, where $i, j \in \{1, 3, 4\}$
	\item \textbf{Finalization:} $x_i$, where $i \in \{1, 3, 4\}$, commits to $T$
\end{enumerate}

\section*{References}
\beginrefs
\bibentry{1}{\sc Jian Liu} and {\sc Wenting Li}, and {\sc Ghassan O. Karame}, and {\sc N. Asokan}
``Toward Fairness of Cryptocurrency Payments,''
{\it IEEE S\&P},
2018.
\endrefs

% **** THIS ENDS THE EXAMPLES. DON'T DELETE THE FOLLOWING LINE:

\end{document}

